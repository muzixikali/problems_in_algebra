\chapter{多项式}
\section{多项式整除}

\begin{knowledge}[多项式环]
    数域$\mathbb{K}$上一元多项式的全体记为$\mathbb{K}[x]$,称为多项式环。
\end{knowledge}

\begin{knowledge}[带余除法]
    设$f(x),g(x)\in \mathbb{K}[x]$,且$g(x)\ne 0$,则存在唯一的$q(x)$及$r(x)\in \mathbb{K}[x]$,使得
    \begin{equation}
        f(x)=q(x)g(x)+r(x)
    \end{equation}

    其中,$r(x)=0$或$\deg r(x)<\deg g(x)$

    若$r(x)=0$,则称$g(x)$整除$f(x)$,记为$g(x)\mid f(x)$
\end{knowledge}

\begin{knowledge}[多项式等价]\label{多项式等价}
    设$f(x),g(x)\in \mathbb{K}[x]$,则
    \begin{equation}
        g(x)\mid f(x)\mbox{且}f(x)\mid g(x) \Longleftrightarrow \exists c\ne 0\in \mathbb{K},\mbox{使得}f(x)=cg(x)
    \end{equation}
\end{knowledge}

\begin{knowledge}[整除的传递性] \label{整除的传递性}
    若$f(x),g(x),h(x)\in \mathbb{K}$,满足$f(x)\mid g(x)$且$g(x)\mid h(x)$,则$f(x)\mid h(x)$
\end{knowledge}

\begin{knowledge}[整除线性组合]
    \label{整除线性组合}
    设$g(x)|f_{i}(x)(i=1,2,\cdots,s)$,则对任意的$u_i(x)\in \mathbb{K}[x]$,有$g(x)\mid \overset{s}{\underset{i=1}{\sum}}u_i(x)f_i(x)$
\end{knowledge}

\begin{knowledge}[数域扩张不影响整除性]
    多项式的整除性与系数所在数域的扩张无关. 即:

    若数域$\mathbb{K}_1\subset \mathbb{K}_2$,而$f(x),g(x)\in \mathbb{K}_1$,则
    \begin{equation}
        \mbox{在$\mathbb{K}_1[x]$中}f(x)\mid g(x)\Longleftrightarrow \mbox{在$\mathbb{K}_2[x]$中}f(x)\mid g(x)
    \end{equation}
\end{knowledge}

\begin{example}
    设多项式$f(x)=(x+1)^{k+n}+2x(x+1)^{k+n-1}+\cdots +(2x)^k(x+1)^n$,其中$k,n$都是正整数,证明:
    \begin{equation*}
        x^{k+1}\mid [(x-1)f(x)+(x+1)^{k+n+1}]
    \end{equation*}
\end{example}

\begin{proof}
    提公因式,得
    \begin{equation*}
        f(x)=(x+1)^n[(x+1)^k+2x(x+1)^{k-1}+\cdots + (2x)^k]
    \end{equation*}

    注意到公式
    \begin{equation}
        a^{n+1}-b^{n+1}=(a-b)(a^n+ab^{n-1}+\cdots +b^n) \label{1.1.4}
    \end{equation}


    故
    \begin{equation*}
        (1-x)f(x)=[(x+1)-2x]f(x)=(x+1)^n[(x+1)^{k+1}-(2x)^{k+1}]
    \end{equation*}

    则
    \begin{equation*}
        (x-1)f(x)+(x+1)^{k+n+1}=(x+1)^n[(x+1)^{k+1}-(x+1)^{k+1}+(2x)^{k+1}]=(x+1)^n(2x)^{k+1}
    \end{equation*}

    即$(x-1)f(x)+(x+1)^{k+n+1}=[2^{k+1}(x+1)^n]x^{k+1}$,故$x^{k+1}\mid [(x-1)f(x)+(x+1)^{k+n+1}]$得证
\end{proof}

\clearpage

\begin{example}
    设$m,n,p$都是非负整数,证明:$(x^2+x+1)\mid (x^{3m}+x^{3n+1}+x^{3p+2})$
\end{example}

\begin{proof}

    [法1]

    利用公\cref{1.1.4}可知
    \begin{equation*}
        x^{3k}-1=(x^3)^k-1=(x^3-1)[(x^3)^{k-1}+(x^3)^{k-2}+\cdots +1]=(x-1)(x^2+x+1)(x^{3k-3}+x^{3k-6}+\cdots +1)
    \end{equation*}

    故$(x^2+x+1)\mid (x^{3k}-1)$

    而
    \begin{equation*}
        x^{3m}+x^{3n+1}+x^{3p+2}=(x^{3m}-1)+x(x^{3n}-1)+x^2(x^{3p}-1)+(1+x+x^2)
    \end{equation*}

    由\cref{整除线性组合}可得,$(x^2+x+1)\mid (x^{3m}+x^{3n+1}+x^{3p+2})$

    [法2]

    设$\omega_1,\omega_2$是$x^2+x+1$的两个根,则有$\omega_i^2+\omega_i+1=0$

    且$\omega_i^3-1=(\omega_i-1)(\omega_i^2+\omega_i+1)=0$,即$\omega_i^3=1(i=1,2)$

    故
    \begin{equation*}
        \omega_i^{3m}+\omega_i^{3n+1}+\omega_i^{3p+2}=1+\omega_i+\omega_i^2=0
    \end{equation*}

    即
    \begin{equation*}
        (x-\omega_i)\mid x^{3m}+x^{3n+1}+x^{3p+2}
    \end{equation*}

    又$(x-\omega_1,x-\omega_2)=1$,
    故$x^2+x+1=(x-\omega_1)(x-\omega_2)\mid (x^{3m}+x^{3n+1}+x^{3p+2})$
\end{proof}

\begin{example}
    设$f(x),g(x),h(x),p(x)$均为实系数多项式,且满足
    \begin{equation}
        (x^2+1)h(x)+(x-1)f(x)+(x-1)g(x)=0   \label{1.1.5}
    \end{equation}
    \begin{equation}
        (x^2+1)p(x)+(x+1)f(x)+(x+2)g(x)=0   \label{1.1.6}
    \end{equation}

    证明:$f(x),g(x)$均被$x^2+1$整除
\end{example}

\begin{proof}

    [法1]

    \cref{1.1.5}$\times$ (x+2)-\cref{1.1.6}$\times$ (x-1)得
    \begin{equation*}
        (x^2+1)[(x+2)h(x)-(x-1)p(x)]+[(x-1)(x+2)-(x+1)(x-1)]f(x)=0
    \end{equation*}

    即$-(x-1)f(x)=(x^2+1)[(x+2)h(x)-(x-1)p(x)]$

    故$(x^2+1)\mid (x-1)f(x)$

    又$(x-1,x^2+1)=1$,则$(x^2+1)\mid f(x)$

    同理得,$(x^2+1)\mid g(x)$

    [法2]

    在\cref{1.1.5},\cref{1.1.6}中令$x=i$得
    \begin{equation*}
        (i-1)f(i)+(i-1)g(i)=0
    \end{equation*}
    \begin{equation*}
        (i+1)f(i)+(i+2)g(i)=0
    \end{equation*}

    解得$f(i)=0,g(i)=0$,故$(x-i)\mid f(x),(x-i)\mid g(x)$

    同理,在\cref{1.1.5},\cref{1.1.6}中令$x=-i$得
    \begin{equation*}
        (-i-1)f(-i)+(-i-1)g(-i)=0
    \end{equation*}
    \begin{equation*}
        (-i+1)f(-i)+(-i+2)g(-i)=0
    \end{equation*}

    解得$f(-i)=0,g(-i)=0$,故$(x+i)\mid f(x),(x+i)\mid g(x)$

    又$(x+i,x-i)=1$,则
    $x^2+1=(x+i)(x-i)\mid f(x),x^2+1=(x+i)(x-i)\mid g(x)$
\end{proof}

\clearpage

\begin{knowledge}[综合除法]
    设被除式为$f(x)=a_nx^n+a_{n-1}x^{n-1}+\cdots +a_1 x+a_0$

    设除式为$g(x)=x-r$

    设商为$q(x)=b_{n-1}x^{n-1}+\cdots +b_1x+b_0$

    设余数为$s$

    若已知$f(x)$,$g(x)$,可则根据表\ref{综合除法}计算$q(x),s$
\end{knowledge}

\begin{table}[htbp]
    \centering
    \renewcommand{\arraystretch}{1.5}
    \begin{tabular}{c|ccccc}
        & $a_n$ & $a_{n-1}$ & $\cdots$ & $a_1$ & $a_0$ \\ 
        r & & $b_{n-1}r$ & $\cdots$ & $b_1r$ & $b_0$\\ 
        \hline 
        & $a_n$ & $a_{n-1}+b_{n-1}r$ & $\cdots$ & $a_1+b_1r$ & $a_0+b_0r$ \\ 
        & = $b_{n-1}$ & $b_{n-2}$ & $\cdots$ & $b_0$ & s
    \end{tabular}
    \caption{综合除法表格}
    \label{综合除法}
\end{table}

\begin{example}
    利用综合除法将$2x^4+x^3+7x^2-8x+14$表示成$ax(x-1)(x-2)(x-3)+bx(x-1)(x-2)+cx(x-1)+dx+e$,
    求出$a,b,c,d,e$的值(必须列出综合除法算式)
\end{example}

\begin{solution}
    先后用综合除法除以$x,x-1,x-2,x-3$得到余数$e,d,c,b$和最后的商$a$:
\end{solution}

\begin{table*}[htbp]
    \centering
    \renewcommand{\arraystretch}{1.5}
    \begin{tabular}{c|ccccc}
        & 2 & 1 & 7 & -8 & 14\\ 
        0 & & 0 & 0 & 0 & 0 \\ 
        \hline 
        & 2 & 1 & 7 & -8 & e=14 \\
        1 & & 2 & 3 & 10 & \\ 
        \hline
        & 2 & 3 & 10 & d=2 & \\ 
        2 & & 4 & 14 &  \\ 
        \hline 
        & 2 & 7 & c=24 \\ 
        3 & & 6 & \\ 
        \hline
        & a=2 & b=13 & 
    \end{tabular}
\end{table*}\

\vspace{-16pt}
\begin{example}
    设$m$是大于1的正整数,多项式$f(x)=x^{m-1}+x^{m-2}+\cdots +x+1$,试求所有满足$f(x)\mid {[f(x^m)+c]}$的常数$c$
\end{example}

\begin{solution}
    \begin{equation*}
        f(x^m)=(x^m)^{m-1}+(x^m)^{m-2}+\cdots + x^m+1
    \end{equation*}

    当$c=-m$时,
    \begin{equation*}
        f(x^m)-m=((x^m)^{m-1}-1)+((x^m)^{m-2}-1)+\cdots + (x^m-1)
    \end{equation*}
    \begin{equation*}
        =(x^m-1)[\overset{m-2}{\underset{i=0}{\sum}}(x^m)^i+\overset{m-3}{\underset{i=0}{\sum}}(x^m)^i+\cdots
        1]
    \end{equation*}

    故$(x^m-1)\mid [f(x^m)-m]$,又$f(x)\mid (x^m-1)$,则由\cref{整除的传递性}得,$f(x)\mid [f(x^m)-m]$

    \vspace{4pt}
    另一方面,若常数c满足$f(x)\mid [f(x^m)+c]$,则
    \begin{equation*}
        f(x)\mid \{[f(x^m)-m]+(m+c)\}
    \end{equation*}

    因为$f(x)\mid [f(x^m)-m]$,所以$f(x)\mid (m+c)$

    由于$\deg f(x)=m-1>0$,故只有$(m+c)=0$,即$c=-m$时才合理

    \vspace{4pt}
    综上,$c=-m \Longleftrightarrow f(x)\mid {[f(x^m)+c]}$,即满足$f(x)\mid {[f(x^m)+c]}$的常数$c$只有$-m$一个
\end{solution}

\begin{example} \label{例题1.1.6}
    证明:$(x^d-1)\mid (x^n-1)$的充分必要条件是$d\mid n$
\end{example}

\begin{proof}

    设$n=md+r$,其中$0\le r <d$,则
    \begin{equation*}
        x^n-1=x^{md+r}-1=x^r[(x^d)^m-1]+x^r-1=x^r(x^d-1)[(x^d)^{m-1}+\cdots +1]+(x^r-1)
    \end{equation*}

    故$(x^d-1)\mid (x^n-1) \Longleftrightarrow (x^d-1)\mid (x^r-1)\Longleftrightarrow x^r-1=0\Longleftrightarrow r=0 \Longleftrightarrow n=md$

    即$d\mid n$
\end{proof}

\begin{example} \label{例题1.1.7}
    证明:$(x^m-1,x^n-1)=x^d-1\Longleftrightarrow (m,n)=d$
\end{example}

\begin{proof}
    令$h(x)=(x^m-1,x^n-1)$

    【$(m,n)=d \Longrightarrow h(x)=x^d-1$】 

    若$(m,n)=d$,则$\exists u,v\in \mathbb{Z}$,使得$d=mu+nv$,

    若($u>0,v>0$)或($u<0,v<0$),则($d>m$且$d>n$)或(d<0),均不合理

    显然,$uv<0$,不妨设$u>0,v<0$

    由\cref{例题1.1.6}得,
    $$d\mid m,d\mid n \Longleftrightarrow (x^d-1)\mid (x^m-1),(x^d-1)\mid (x^n-1)$$

    故$(x^d-1)\mid h(x)$

    另一方面,$x^{mu}-1=x^{d-nv}=x^{-nv}(x^d-1)+(x^{-nv}-1)$

    而由\cref{整除的传递性}得$h(x)\mid (x^m-1)\mid (x^{mu}-1),h(x)\mid (x^n-1) \mid (x^{-nv}-1)$

    则$h(x)\mid x^{-nv}(x^d-1)$,而$(h(x),x^{-nv})=1$,
    
    故$h(x)\mid (x^d-1)$

    综上,由\cref{多项式等价}得,$h(x)=c(x^d-1)$

    又$h(x)$首项系数为1,故$c=1$,即$h(x)=x^d-1$

    【$ h(x)=x^d-1 \Longrightarrow (m,n)=d$】

    若$h(x)=x^d-1$,令$(m,n)=e$

    则由已证可知,$h(x)= x^e-1$

    故$x^d-1=x^e-1\Longleftrightarrow d=e$,即$(m,n)=d$
\end{proof}

\begin{example}
    设$m,n$是两个大于1的正整数,多项式
    \begin{equation*}
        f(x)=x^{m-1}+x^{m-2}+\cdots + x + 1
    \end{equation*}
    \begin{equation*}
        g(x)=x^{n-1}+x^{n-2}+\cdots + x + 1
    \end{equation*}

    证明:$(f(x),g(x))=1 \Longleftrightarrow (m,n)=1$
\end{example}

\begin{proof}

    [法1]

    $x^m-1=(x-1)f(x)$,$x^n-1=(x-1)g(x)\xlongleftrightarrow{(f(x),g(x))=1} (x^m-1,x^n-1)=x-1$

    根据\cref{例题1.1.6},有$(x^m-1,x^n-1)=x-1\Longleftrightarrow (m,n)=1$

    [法2]

    【$ (m,n)=1 \Longrightarrow (f(x),g(x))=1$】

    设$\omega$是一个$n$次本原单位根,即$\omega^n=1$,则$\omega,\omega^2,\cdots \omega^{n-1}$是$g(x)$的全部根

    对于$1\le k\le n-1$,若$f(\omega^k)=0$,则由$x^m-1=(x-1)f(x)$可知,$(\omega^k)^m=1$,

    从而$n\mid km$,这与$(m,n)=1$矛盾,这说明$g(x)$的根都不是$f(x)$的根,即$(f(x),g(x))=1$

    【$ (f(x),g(x))=1 \Longrightarrow  (m,n)=1$】

    设$(m,n)=d>1$,且$m=m_1d,n=n_1d$,易知$1\le n_1<n$

    由于$(\omega^{n_1})^m=(\omega)^{m_1n_1d}=(\omega^n)^{m_1}=1$,而$\omega^{n_1}\ne 1$,

    所以$\omega^{n_1}$也是$f(x)$的根

    这与$(f(x),g(x))=1$产生矛盾,故$(m,n)=1$
\end{proof}

\clearpage
\begin{example}
    设$f(x)=x^7+2x^6-6x^5-8x^4+19x^3+9x^2-22x+8$,$g(x)=x^2+x-2$,将$f(x)$表示成$g(x)$的方幂和,即将$f(x)$表示成
    \begin{equation*}
        f(x)=c_k(x)[g(x)]^k + c_{k-1}(x)[g(x)]^{k-1}+ \cdots +c_1(x)g(x)+c_0(x)
    \end{equation*}
    其中,$\deg c_i(x)<\deg g(x)$或$c_i(x)=0$\quad $i=0,1,\cdots,k$
\end{example}

\begin{solution}
    
    以$g(x)$为除式,对$f(x)$持续做带余除法,得到余式$c_0(x),c_1(x),\cdots,c_{k-1}(x)$和最后的商$c_k(x)$
    \begin{equation*}
        f(x)=(x^5+x^4-5x^3-x^2+10x-3)g(x)+(x+2)
    \end{equation*}
    \begin{equation*}
        (x^5+x^4-5x^3-x^2+10x-3)=(x^3-3x+2)g(x)+(2x+1)
    \end{equation*}
    \begin{equation*}
        (x^3-3x+2)=(x-1)g(x)+0
    \end{equation*}

    故$c_0(x)=x+2,c_1(x)=2x+1,c_2(x)=0,c_3(x)=x-1$,

    即$f(x)=(x-1)[g(x)]^3+(2x+1)g(x)+(x+2)$
\end{solution}

\begin{example}\label{例题1.1.10}
    假设$f_0(x^5)+xf_1(x^{10})+x^2f_2(x^{15})+x^3f_3(x^{20})$能被$x^4+x^3+x^2+x+1$整除,证明:$f_i(x)(i=0,1,2,3)$能被$x-1$整除 
\end{example}

\begin{proof}
    
    设$\omega_k$为5次单位根,即
    \begin{equation*}
        w_k=\cos \frac{2k\pi}{5}+i\sin\frac{2k\pi}{5}\quad k=0,1,\cdots,4
    \end{equation*}

    则$\omega_k^5=1$,且$x^4+x^3+x^2+x+1=\prod \limits_{k=1}^{4}(x-\omega_k)$

    所以
    \begin{equation*}
        (x-\omega_k)\mid (x^4+x^3+x^2+x+1) \mid \sum\limits_{j=0}^{3}x^jf_j(x^{5(j+1)})\quad k=1,2,3,4
    \end{equation*}

    由此得到,
    \begin{equation}\label{方程组1}
        \begin{cases}
            f_0(1)+\omega_1f_1(1)+\omega_1^2f_2(1)+\omega_1^3f_3(1)=0,  \\ 
            f_0(1)+\omega_2f_1(1)+\omega_2^2f_2(1)+\omega_2^3f_3(1)=0,  \\ 
            f_0(1)+\omega_3f_1(1)+\omega_3^2f_2(1)+\omega_3^3f_3(1)=0,  \\ 
            f_0(1)+\omega_4f_1(1)+\omega_4^2f_2(1)+\omega_4^3f_3(1)=0,  \\ 
        \end{cases}
    \end{equation}

    该方程为关于$f_i(1)(i=0,1,2,3)$的齐次方程,且其系数行列式
    $       d=\begin{bmatrix}
            1 & \omega_1 & \omega_1^2 & \omega_1^3 \\ 
            1 & \omega_2 & \omega_2^2 & \omega_2^3 \\ 
            1 & \omega_3 & \omega_3^2 & \omega_3^3 \\ 
            1 & \omega_4 & \omega_4^2 & \omega_4^3 \\ 
            \end{bmatrix}$为范德蒙德行列式,

    由于$\omega_1\ne \omega_2 \ne \omega_3 \ne \omega_4$,故$d \ne 0\Longrightarrow$方程组\ref{方程组1}只有零解:$f_i(1)=0(i=0,1,2,3)$

    即$f_i(x)(i=0,1,2,3)$能被$x-1$整除 
\end{proof}

\begin{example}
    设$\sum\limits_{i=0}^{n-1}x^iP_i(x^{in})=P(x^n)$,且$(x-1)\mid P(x)$,其中$P_i(x)(0\le i\le n-1)$,$P(x)$均为实系数多项式.证明:

    (1)$P_{i}(x)=0,1\le i\le n-1$;\quad(2)$P(x)=0$\quad(3)$P_0(1)=0$
\end{example}

\begin{proof}
    \begin{equation*}
        (x-1)\mid P(x) \Longrightarrow P(1)=0
    \end{equation*}

    设$\omega_j$为$n$次单位根,即$\omega_j^n=1,j=0,1,\cdots,n-1$

    由于$P(\omega_j^n)=P(1)=0$,将$\omega_j$代入$\sum\limits_{i=0}^{n-1}x^iP_i(x^{in})=P(x^n)$中,得
    \begin{equation*}
        P_0(1)+\omega_jP_1(1)+\omega_j^2P_2(1)+\cdots +\omega_j^{n-1}P_{n-1}(1)=0\quad 0\le j\le n-1
    \end{equation*}

    同\cref{例题1.1.10},可解得$P_0(1)=P_1(1)=\cdots=P_{n-1}(1)=0$

    设$P_{n-1}(x)=\sum\limits_{t=0}^s a_tx^t$,任取其中一项$a_m x^m(m\ge 1)$,考虑其在$x^{n-1}P_{n-1}(x^{(n-1)n})$中的对应项:
    \begin{equation}\label{1.1.8}
        x^{n-1}a_m (x^{(n-1)n})^m=a_mx^{n-1+m(n-1)n}
    \end{equation}

    \vspace{4pt}
    对于$0\le i <n-1$,易知$x^i P_i(x^{in})$都不含\cref{1.1.8}的同类项:

    事实上,若某个$x^i P_i(x^{in})$含$cx^{i+kin}$,其中$k$是$P_i(x)$中某个方幂$x^k$的次数,则
    \begin{equation*}
        i+kin=n-1+m(n-1)n\Longrightarrow \frac{n-1-i}{n}=ik-m(n-1)
    \end{equation*}

    真分数=整数,产生矛盾

    \vspace{4pt}
    同理,$P(x^n)$中也不含\cref{1.1.8}的同类项:

    否则,存在$P(x)$中的某个方幂$x^k$,使得
    \begin{equation*}
        kn=n-1+m(n-1)n\Longrightarrow \frac{n-1}{n}=k-m(n-1)
    \end{equation*}

    真分数=正整数,产生矛盾

    综上,\cref{1.1.8}的系数$a_m=0$,又$P_{n-1}(1)=0\Longrightarrow a_0=0$,

    即$P_{n-1}(x)=0$

    利用同样的证法可证,$P_1(x)=P_2(x)=\cdots P_{n-2}(x)=0$,故$P(x)=0$. 第(1),(2),(3)问均得证
\end{proof}

\begin{example}
    设$f(x)$是一个次数大于0的复系数多项式,且满足条件$f(f(x))=[f(x)]^n$,这里$n$是某一个正整数.证明:$f(x)=x^n$
\end{example}

\begin{proof}
    设$\deg f(x)=k>0$,则可设$f(x)=a_k x^k+\cdots + a_1 x+a_0$,其中$a_k\ne 0$

    则$f(f(x))=a_k [f(x)]^k+\cdots + a_1 f(x)+a_0$

    故$\deg f(f(x))=\deg [f(x)]^k=k^2$

    又$\deg [f(x)]^n=nk$,则$k^2=nk\Longrightarrow k=n$

    则$f(f(x))=a_n [f(x)]^n+\cdots + a_1 f(x)+a_0$

    又$f(f(x))=[f(x)]^n\Longrightarrow a_n=1,a_{n-1}=\cdots a_0=0$

    即$f(x)=x^n$
\end{proof}

\begin{example}
    设$f(x)\in F[x]$是一个次数大于0的多项式.证明:$f(x)$不可约的充分必要条件是:由“$f(x)$整除两个两个多项式的乘积”可推出“$f(x)$必整除其中的一个”. 
\end{example}

\begin{proof}
    [必要性]若$f(x)$在$F$上不可约,且$f(x)\mid (g(x)h(x))$,其中$g(x),h(x)\in F[x]$. 

    如果$f(x)\nmid g(x)$,那么$(f(x),g(x))=1$,必有$f(x)\mid h(x)$

    [充分性]设由$f(x)\mid ((g(x),h(x)))$可推出$f(x)\mid g(x)$或$f(x)\mid h(x)$,下证$f(x)$在$F$上不可约. 

    事实上,若$f(x)$在$F$上可约,则$\exists a_1(x),a_2(x)\in F[x]$,使得$f(x)=a_1(x)a_2(x)$,
    
    其中,$0<\deg a_i(x)< \deg f(x)\quad i=1,2$

    于是$f(x)\mid (a_1(x)a_2(x))\Longrightarrow f(x)\mid a_1(x)$或$f(x)\mid a_2(x)$

    次数产生矛盾. 故$f(x)$在$F$上不可约. 
\end{proof}

\begin{example}
    设$f(x)$是数域$\mathbb{K}$上的一个次数大于0的一元多项式.证明:$f(x)$是一个不可约多项式$p(x)$的幂(即$\exists n\in \mathbb{N}^+$,使得$f(x)=p^m(x))$的充分必要条件是,对任意的多项式$g(x)$和$h(x)$,若$f(x)\mid (g(x)h(x))$,则必有$f(x)\mid g(x)$或$f(x)\mid h^n(x)$,其中$n$是某一个正整数.
\end{example}

\begin{proof}
    [必要性]设$f(x)=p^m(x)$,且对任意$g(x),h(x)\in\mathbb{K}[x]$,都有$f(x)\mid (g(x)h(x))$. 
    
    若$f(x)\nmid g(x)$,则$f(x)$与$h(x)$不互素,则$p(x)\mid h(x)\Longrightarrow p(x)^m \mid h(x)^m$,令$n=m$,则$f(x)\mid h^n(x)$

    [充分性]反证法.假设对任意不可约多项式$p(x)$及正整数$m$,有$f(x)=p^m(x)h(x)$,
    
    其中$\deg h(x)>0$,且$p(x)\nmid h(x)$.则$f(x)$不是任何一个不可约多项式$p(x)$的幂. 
    
    取$g(x)=p^m(x)$,则$f(x)\mid ((g(x)h(x)))$,但$f(x)\nmid g(x)$

    则$\exists n\in N^+,$使得$f(x)\mid h^n(x)$,故$p(x)\mid f(x)\mid h^n(x)$

    又$p(x)$为不可约多项式,则必有$p(x)\mid h(x)$,产生矛盾.$f(x)$必是某个不可约多项式的幂.
\end{proof}
