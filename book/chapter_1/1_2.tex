\section{最大公因式与互素}

\begin{knowledge}[线性表示最小公因式]
    对于$\mathbb{K}[x]$中的任意两个多项式$f(x),g(x)$,在$\mathbb{K}[x]$中一定存在最大公因式$d(x)$,

    且存在$u(x),v(x)\in \mathbb{K}[x]$使得$d[x]=u(x)f(x)+v(x)g(x)$
\end{knowledge}

\begin{knowledge}[互素的充要条件]
    多项式$f(x),g(x)\in \mathbb{K}[x]$互素的充分必要条件是存在$u(x),v(x)\in \mathbb{K}[x]$,使得
    \begin{equation}
        u(x)f(x)+v(x)g(x)=1
    \end{equation}
\end{knowledge}

\begin{knowledge}[B$\acute{e}$zout(裴蜀)定理]\label{裴蜀定理}
    设$f(x),g(x)\in \mathbb{K}[x]$的次数分别为$m,n$,且$(f(x),g(x))=1$,则存在唯一的$u(x),v(x)\in \mathbb{K}[x]$,使得
    \begin{equation*}
        u(x)f(x)+v(x)g(x)=1,\mbox{且}\deg u(x)\le n-1,\deg v(x)\le m-1
    \end{equation*}
\end{knowledge}

\begin{knowledge}[互素与整除1]
    设$f(x)\mid (g(x)h(x))$,且$(f(x),g(x))=1$,则$f(x)\mid h(x)$
\end{knowledge}

\begin{knowledge}[互素与整除2]
    设$f_1(x)\mid g(x),f_2(x)\mid g(x)$,且$(f_1(x),f_2(x))=1$,则$(f_1(x)f_2(x))\mid g(x)$
\end{knowledge}

\begin{knowledge}[不可约多项式与整除1]
    设$p(x)\in \mathbb{K}[x]$是不可约多项式,则对$f(x)\in \mathbb{K}[x],(p(x),f(x))=1$或$p(x)\mid f(x)$
\end{knowledge}

\begin{knowledge}[不可约多项式与整除2]
    设$p(x)\in \mathbb{K}[x]$是不可约多项式,$f(x),g(x)\in \mathbb{K}[x]$,则由$p(x)\mid (f(x),g(x))$一定能推出$p(x)\mid f(x)$或$p(x)\mid g(x)$
\end{knowledge}

\begin{example}
    设$f(x)=x^4+x^3-3x^2-4x-1$,$g(x)=x^3+x^2-x-1$,求$u(x),v(x)$使$u(x)f(x)+v(x)g(x)=(f(x),g(x))$
\end{example}

\begin{solution}
    直接使用辗转相除法:
    \begin{center}
        \begin{tabular}{c|c|c|c}
            $q_2(x)=$ & $g(x)=x^3+x^2-x-1$ & $f(x)=x^4+x^3-3x^2-4x-1$ & $x=q_1(x)$ \\ 
            $-\frac{1}{2}x+\frac{1}{4}$ & \quad \quad$x^3+\frac{3}{2}x^2+\frac{1}{2}x$ & $x^4+x^3-x^2-x$ & \\ 
            \hline
            & $-\frac{1}{2}x^2-\frac{3}{2}x-1$ & $r_1(x)=-2x^2-3x-1$ & $\frac{8}{3}x+\frac{4}{3}$ \\ 
            & $-\frac{1}{2}x^2-\frac{3}{4}x-\frac{1}{4}$ & $-2x^2-2x$ & $=q_3(x)$ \\ 
            \hline 
            & $r_2(x)=-\frac{3}{4}x-\frac{3}{4}$ & $-x-1$ & \\ 
            & & $-x-1$ & \\ 
            \hline
            & & $r_3(x)=0$
        \end{tabular}
    \end{center}
    
    得$(f(x),g(x))=x+1$,且$f(x)=q_1(x)g(x)+r_1(x),g(x)=q_2(x)r_1(x)+r_2(x)$,故
    $$(f(x),g(x))=-\frac{4}{3} r_2(x) = -\frac{4}{3} [g(x)-q_2(x)r_1(x)]=-\frac{4}{3}g(x)+\frac{4}{3}q_2(x)[f(x)-q_1(x)g(x)]$$
    $$=\frac{4}{3}q_2(x)f(x) - \frac{4}{3}[1+q_1(x)q_2(x)]g(x)$$

    取$u(x)=\frac{4}{3}q_2(x)=-\frac{2}{3}x+\frac{1}{3},v(x)=-\frac{4}{3}[1+q_1(x)q_2(x)]=\frac{2}{3}x^2-\frac{1}{3}x-\frac{4}{3}$,

    则$u(x)f(x)+v(x)g(x)=(f(x),g(x))$
\end{solution}

\clearpage
\begin{example}
    设$f(x)=x^3+(1+k)x^2+2x+2l$与$g(x)=x^3+kx^2+l$的最大公因式是一个二次多项式,求$k,l$的值
\end{example}

\begin{solution}
    
    [法1]令$d(x)=(f(x),g(x)),h(x)=f(x)-g(x)=x^2+2x+l$

    则有$\deg d(x)=\deg h(x)=2$,且$d(x)\mid h(x)$,

    又$d(x),h(x)$都是首1多项式,则$d(x)=h(x)$

    令$g(x)=(x-c)d(x)=(x-c)h(x)=(x-c)(x^2+2x+l)$

    比较系数得$\begin{cases}
        2-c=k \\ 
        l-2c=0 \\ 
        -cl=l 
    \end{cases}$
    ,解得$\begin{cases}
        c=0 \\ 
        l=0 \\ 
        k=2
    \end{cases}$
    或$\begin{cases}
        c=-1 \\ 
        l=-2 \\ 
        k=3
    \end{cases}$

    [法2]设$(f(x),g(x))=x^2+ax+b$,则$f(x)=(x-p)(x^2+ax+b),g(x)=(x-q)(x^2+ax+b)$

    比较系数得$\begin{cases}
        1+k = a-p \\ 
        2 = b-ap \\ 
        2l = -bp \\ 
        k = a-q \\ 
        0 = b-aq \\ 
        l = -bq 
    \end{cases}$,
    解得
    $\begin{cases}
        a=2 \\
        b=0 \\ 
        l=0 \\ 
        k=2 \\ 
        p=-1 \\ 
        q=0
    \end{cases}$或
    $\begin{cases}
        a=2 \\
        p=-2 \\
        q=-1 \\ 
        b=-2 \\ 
        k=3 \\ 
        l=-2
    \end{cases}$
\end{solution}

\begin{example}
    设$f(x),g(x)\in \mathbb{F}[x],a,b,c,d\in \mathbb{F}$,且$\begin{bmatrix}
        a & b \\ c & d
    \end{bmatrix} \ne 0$. 证明:$(f(x),g(x))=(af(x)+bg(x),cf(x)+dg(x))$
\end{example}

\begin{proof}

    设$\begin{cases}
        h_1(x)=af(x)+bg(x) \\ 
        h_2(x)=cf(x)+dg(x)
    \end{cases}$,并记$d(x)=(f(x),g(x)),d_1(x)=(h_1(x),h_2(x))$

    则$d(x)\mid h_1(x),d(x)\mid h_2(x)\Longrightarrow d(x)\mid d_1(x)$

    另一方面,由于$\begin{bmatrix}
        a & b \\ c & d
    \end{bmatrix} =ad-bc \ne 0$
    ,所以可以解得
    $\begin{cases}
        f(x)=\frac{d}{ad-bc}h_1(x)-\frac{b}{ad-bc}h_2(x) \\ 
        g(x)=-\frac{c}{ad-bc}h_1(x)+\frac{a}{ad-bc}h_2(x)
    \end{cases}$

    故$ d_1(x)\mid f(x),d_2(x)\mid g(x) \Longrightarrow d_1(x)\mid d(x)$

    又$d(x),d_1(x)$都是首1多项式,故$d(x)=d_1(x)$,即$(f(x),g(x))=(h_1(x),h_2(x))$
\end{proof}

\begin{example}\label{例题1.2.4}
    设$f(x),g(x)\in \mathbb{P}[x]$,$d(x)=(f(x),g(x))$,且$\deg \frac{f(x)}{d(x)}\ge 1$,$\deg \frac{g(x)}{d(x)}\ge 1$,则存在唯一的$u(x),v(x)\in \mathbb{P}[x]$使得$u(x)f(x)+v(x)g(x)=d(x)$,其中$\deg u(x)<\deg \frac{g(x)}{d(x)},\deg v(x)<\frac{f(x)}{d(x)}$
\end{example}

\begin{proof}
    
    令$f_1(x)=\frac{f(x)}{d(x)}$,$g_1(x)=\frac{g(x)}{d(x)}$,则只需证:存在唯一的$u(x),v(x)\in \mathbb{P}[x]$使得$u(x)f_1(x)+v(x)g_1(x)=1$,
    
    其中$\deg u(x)<\deg g_1(x),\deg v(x)<\deg f_1(x)$ 

    [存在性]易知$(f_1(x),g_1(x))=1$,则存在$u_1(x),v_1(x)\in \mathbb{P}[x]$,使得
    \begin{equation}\label{式1.2.2}
        u_1(x)f_1(x)+v_1(x)g_1(x)=1
    \end{equation}

    注意到$\deg f_1(x)\ge 1,\deg g_1(x)\ge 1$,所以$f_1(x)\nmid v_1(x)$且$g_1(x)\nmid u_1(x)$

    由带余除法得,存在唯一的$u(x),v(x)\in \mathbb{P}[x]$,使得
    \begin{equation}\label{式1.2.3}
        u_1(x)=p(x)g_1(x)+u(x) \quad \deg u(x) < \deg g_1(x)
    \end{equation}
    \begin{equation}\label{式1.2.4}
        v_1(x)=q(x)f_1(x)+v(x) \quad \deg v(x) < \deg f_1(x)
    \end{equation}

    将\cref{式1.2.3},\cref{式1.2.4}代入\cref{式1.2.2}得,
    $$u(x)f_1(x)+v(x)g_1(x)+[p(x)+q(x)]f_1(x)g_1(x)=1$$

    比较次数得,$p(x)+q(x)=0$,于是$u(x)f_1(x)+v(x)g_1(x)=1$

    \clearpage
    [唯一性]若另存在$u_2(x),v_2(x)\in \mathbb{P}[x]$,使得
    \begin{equation}\label{式1.2.5}
        u_2(x)f_1(x)+v_2(x)g_1(x)=1,\mbox{且}
        \deg u_2(x)<\deg g_1(x),\deg v_2(x)<\deg f_1(x)
    \end{equation}

    \cref{式1.2.2}与\cref{式1.2.5}相减得,
    \begin{equation}\label{式1.2.6}
        [u(x)-u_2(x)]f_1(x)+[v(x)-v_2(x)]g_1(x)=0
    \end{equation}

    于是$g_1(x)\mid [u(x)-u_2(x)]f_1(x)$.由于$(f_1(x),g_1(x))=1$,故$g_1(x)\mid [u(x)-u_2(x)]$

    但$\deg [u(x)-u_2(x)]<\deg g_1(x)$,故$u(x)-u_2(x)=0$,即$u(x)=u_2(x)$

    又由\cref{式1.2.6}及$\deg g_1(x)\ge 1$得$v(x)-v_2(x)=0\Longrightarrow v(x)=v_2(x)$,
\end{proof}

\begin{example}
    求多项式$f(x),g(x)$,使得
    \begin{equation}
        x^mf(x)+(1-x)^ng(x)=1   \label{1.2.7}
    \end{equation}
\end{example}

\begin{solution}
    
    由于$x^m$与$(1-x)^n$互素,合适的$f(x),g(x)$一定存在

    假定$\deg f(x)\le n-1,\deg g(x)\le m-1$,则由\cref{裴蜀定理}或\cref{例题1.2.4}得,$f(x),g(x)$唯一

    不妨令$f(x)=\sum\limits_{i=0}^{n-1}a_i(1-x)^i=\sum\limits_{i=0}^{n-1}(-1)^ia_i(x-1)^i$,$g(x)=\sum\limits_{j=0}^{m-1}b_jx^j$,则
    $$a_0=f(1)=1,a_i=\frac{(-1)^i}{i!}f^{(i)}(1),b_0=g(0)=1,b_j=\frac{1}{j!}g^{(j)}(0)\quad i=1,\cdots,n-1\quad j=1,\cdots,m-1$$

    利用莱布尼兹(Leibniz)公式对\cref{1.2.7}两边求导得
    \begin{equation}\label{1.2.8}
        \sum\limits_{i=0}^{k}C_k^i\frac{m!}{[m-(k-i)]!}x^{m-(k-i)}f^{(i)}(x)+\sum\limits_{i=0}^{k}(-1)^{k-i}C_k^i\frac{n!}{[n-(k-i)]!}(1-x)^{n-(k-i)}g^{(i)}(x)=0
    \end{equation}

    设$1\le k \le n-1 \Longrightarrow n-k+i\ge 1,i=0,1,\cdots,k$,在\cref{1.2.8}中令$x=1$得
    $$\sum\limits_{i=0}^{k}C_k^i\frac{m!}{[m-(k-i)]!}f^{(i)}(1)
    =\sum\limits_{i=0}^{k}\frac{k!}{i!(k-i)!}\frac{m!}{[m-(k-i)]!}\cdot (-1)^ii! a_i
    =k!\cdot \sum\limits_{i=0}^{k}(-1)^i C_m^{k-i} a_i
    =0 $$ 
    $$\Longrightarrow \sum\limits_{i=0}^{k}(-1)^i C_m^{k-i} a_i=0$$

    对$k=1,\cdots,n-1$展开解得
    $a_i=C_{m+i-1}^i,i=0,1,2,\cdots,n-1$

    同理,设$1\le k \le m-1 \Longrightarrow m-k+i\ge 1,i=0,1,\cdots,k$,在\cref{1.2.8}中令$x=0$得
    $$\sum\limits_{i=0}^{k}(-1)^{k-i} C_k^i\frac{n!}{[n-(k-i)]!}g^{(i)}(0)
    =\sum\limits_{i=0}^{k}(-1)^{k-i} \frac{k!}{i!(k-i)!}\frac{n!}{[n-(k-i)]!}\cdot i! b_i
    =k!\cdot \sum\limits_{i=0}^{k}(-1)^{k-i} C_n^{k-i} a_i
    =0 $$ 
    $$\Longrightarrow \sum\limits_{i=0}^{k}(-1)^{k-i} C_n^{k-i} a_i=0$$

    对$k=1,\cdots,m-1$展开解得
    $b_j=C_{n+j-1}^j,j=0,1,2,\cdots,m-1$

    综上,$f(x)=\sum\limits_{i=0}^{n-1}C_{m+i-1}^i(1-x)^i,g(x)=\sum\limits_{j=0}^{m-1}C_{n+j-1}^jx^j$
\end{solution}

\begin{example}
    设$$f(x)=a_0+a_1x+a_2x^2+a_{10}x^{10}+a_{11}x^{11}+a_{12}x^{12}+a_{13}x^{13}\quad a_{13}\ne 0$$
    $$g(x)=b_0+b_1x+b_2x^2+b_3x^3+b_{11}x^{11}+b_{12}x^{12}+b_{13}x^{13}\quad b_3\ne 0$$
    是两个复系数多项式. 证明它们的最大公因式的次数最多为6
\end{example}

\begin{proof}
    
    设$d(x)=(f(x),g(x))$,则$d(x)\mid f(x),d(x)\mid g(x)$. 

    将$f(x),g(x)$表示为$$f(x)=f_1(x)+x^{10}f_2(x),g(x)=g_1(x)+x^{10}g_2(x)$$

    其中$$f_1(x)=a_0+a_1x+a_2x^2,f_2(x)=a_{10}+a_{11}x+a_{12}x^2+a_{13}x^3$$
    $$g_1(x)=b_0+b_1x+b_2x^2+b_3x^3,g_2(x)=b_{11}x+b_{12}x^2+b_{13}x^3$$

    注意到$$f(x)g_2(x)-g(x)f_2(x)=f_1(x)g_2(x)+x^{10}f_2(x)g_2(x)-g_1(x)f_2(x)-x^{10}g_2(x)f_2(x)=f_1(x)g_2(x)-g_1(x)f_2(x)$$

    故$d(x)\mid [f_1(x)g_2(x)-g_1(x)f_2(x)]$,又$a_{13}b_3\ne 0$,则$\deg [f_1(x)g_2(x)-g_1(x)f_2(x)]=6$,故$\deg d(x)\le 6$
\end{proof}

\begin{example}
    设$f(x),g(x)\in \mathbb{F}[x]$,且$f(x)\ne 0$. 

    (1)证明:若$(f(x),g(x))=1$,则对任意的$h(x)\in \mathbb{F}[x]$,有$(f(x),g(x)h(x))=(f(x),h(x))$

    (2)问:若存在$h(x)\in \mathbb{F}[x]$,满足$(f(x),g(x)h(x))=(f(x),h(x))$,则是否一定有$(f(x),g(x))=1$?为什么?

    (3)若对任意$h(x)\in \mathbb{F}[x]$,满足$(f(x),g(x)h(x))=(f(x),h(x))$,则是否一定有$(f(x),g(x))=1$?为什么?
\end{example}

\begin{solution}

    (1)因$(f(x),g(x))=1$,故存在$u(x),v(x)\in \mathbb{F}[x]$,使得$u(x)f(x)+v(x)g(x)=1$

    从而$u(x)f(x)h(x)+v(x)g(x)h(x)=h(x)$

    若$a(x)\in \mathbb{F}[x]$满足$a(x)\mid f(x),a(x)\mid g(x)h(x)$,则$a(x)\mid h(x)$,即$f(x)$与$g(x)h(x)$的公因式一定是$f(x)$与$h(x)$的公因式,反之显然也有$f(x)$与$h(x)$的公因式一定是$f(x)$与$g(x)h(x)$的公因式
    
    故$f(x),g(x)h(x)$和$f(x),h(x)$有完全相同的公因式,则必有$(f(x),g(x)h(x))=(f(x),h(x))$

    (2)不一定. 如$f(x)=(x+1)(x-1),g(x)=x+1,h(x)=(x+1)(x-1)^2$

    满足$(f(x),g(x)h(x))=(f(x),h(x))=(x+1)(x-1)$,但$(f(x),g(x))=x+1\ne 1$

    (3)是. 对于$h(x)=1$,有$(f(x),g(x))=(f(x),1)=1$
\end{solution}

\begin{example}
    设$M$为$\mathbb{F}[x]$中一切形如$u(x)f(x)+v(x)g(x)$的非零多项式所构成的集合,其中$f(x),g(x)$是$\mathbb{F}[x]$中给定的非零多项式,$u(x),v(x)$是$\mathbb{F}[x]$中任意的多项式.证明:$M$非空,且$M$中次数最低的多项式都是$f(x),g(x)$的最大公因式. 
\end{example}

\begin{proof}

    $M=\{u(x)f(x)+v(x)g(x)|u(x),v(x)\in \mathbb{F}[x]\}$.由于$f(x),g(x)\in M\Longrightarrow M$非空

    令$d(x)=(f(x),g(x))$,则存在$u_0(x),v_0(x)\in \mathbb{F}[x]$,使得$d(x)=u_0(x)f(x)+v_0g(x)$,可见$d(x)\in M$

    现任取$h(x)\in M$,且次数最低,则有$\deg h(x)\le \deg d(x)$

    又$h(x)=u(x)f(x)+v(x)g(x)$,则$d(x)\mid h(x)\Longrightarrow \deg d(x)\le \deg h(x)$

    故$\deg d(x)\le \deg h(x) \le \deg d(x)$,可见$\deg h(x)=\deg d(x)\Longrightarrow h(x)=cd(x)$,即$M$中次数最低的多项式$h(x)$都是$f(x),g(x)$的最大公因式 
\end{proof}

\begin{example}
    设$F$是一数域,多项式$f(x),g(x)\in \mathbb{F}[x]$具有性质:当$h(x)\in \mathbb{F}[x]$,且$f(x)\mid h(x),g(x)\mid h(x)$时,必有$f(x)g(x)\mid h(x)$.证明:$(f(x),g(x))=1$
\end{example}

\begin{proof}
    
    反证法.假设$(f(x),g(x))=d(x),\deg d(x)>0$,则$f(x)=d(x)u(x),g(x)=d(x)v(x)$,
    
    其中$(u(x),v(x))=1$,且有$\deg u(x)<\deg f(x),\deg v(x)<\deg g(x)$

    则$f(x)\mid f(x)v(x) = u(x)d(x)v(x)=u(x)g(x)$,
    
    又$g(x)\mid u(x)g(x)$,则根据题设,必有$f(x)g(x)\mid u(x)g(x)$.这与$\deg f(x)g(x)>\deg u(x)g(x)$矛盾. 

    则$(f(x),g(x))=1$
\end{proof}